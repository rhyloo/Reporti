% Created 2025-06-01 Sun 21:43
% Intended LaTeX compiler: xelatex
\documentclass{reporti}
\date{}
\title{Reporti: TI-Inspired \LaTeX{} Template.}
\hypersetup{
 pdfauthor={J. L. Benavides},
 pdftitle={Reporti: TI-Inspired \LaTeX{} Template.},
 pdfkeywords={},
 pdfsubject={},
 pdfcreator={Emacs 29.3 (Org mode 9.6.15)}, 
 pdflang={Spanish}}
\begin{document}

\subtitle{LaTeX Template}
\author{Rhyloo - \href{mailto:jorge2@uma.es}{jorge2@uma.es}}
\logo{rhyloo_solutions_horizontal.pdf}

\type{README}
\projectcode{REPORTI_DOC_TEMPLATE_V0.0}

\summary{}

\notes{PurePath Digital, Equibit are trademarks of Texas Instruments.\\ Excel is a trademark of Microsoft Corporation.}

\cover[width=1.35\textwidth][continue]


\section{Motivation}
\label{sec:orgfba59e4}

When I was at college I had to read a lot of datasheets and documentation from different manufacturers of electronics (i. e. Texas Instruments, Analog Devices, Microchip and others\ldots{}). My favorite and inspiration/copy is Texas Instruments.

I really enjoy reading the docs from them!

I would like to understand how their docs are made, which is the department that do them and why they are not using \LaTeX{}\ldots{} 

\section{Features}
\label{sec:org50fb53f}

\subsection{Cover page}
\label{sec:org89b5201}
Just trying to keep it simple. A .cls file based on report class.

About features, the cover can be customize with the next commands: Check \ref{lst:org684c652}

\begin{listing}[htbp]
\begin{minted}[]{latex}
\type{Name of doc type}
\code{Project Number/Code} % For more info. Look for naming conventions
\titulo{Project title} 
\autor{Author}
\subtitulo{Subtitle}
\resumen{Summary}
\notes{Some notes in Cover}
\contact{Contact footnote}
\end{minted}
\caption{\label{lst:org684c652}Config commands}
\end{listing}

\subsection{Table}
\label{sec:orgab6ea81}

\subsection{Images}
\label{sec:orgc3633fe}


\section{About images}
\label{sec:orgbadd380}
The file \texttt{Logo-EII-horizontal.pdf} doesn't have any type of LICENSE on the \href{https://www.uma.es/escuela-de-ingenierias-industriales/info/108566/logo-simbolo-de-la-eii/}{web}. As master's student I can use, maybe you don't, be careful with it.

I am really interested in naming conventions applied to projects. I think is the first part of a good story, I define two names the first is common use and the second is like a code, i.e Fat-Man is the bomb, but if I said Fat-Man-DOC-123HAJK123

Do you have one for owns?? Explain in comments and discuss the best :D

Examples:
\begin{itemize}
\item Texas Instruments have names like: TAS5518-5261K2EVM. According to Deepseek
\begin{itemize}
\item TAS: Indicates a Texas Instruments Audio Solution
\item 5518: The specific model number of the IC.
\item 5261: A code unique to the EVM design
\item K2: Likely denotes a revision or variant of the EVM
\item EVM: Standard suffix for Evaluation Module
\end{itemize}
\end{itemize}

IMO is the first part of a good story like "MarkII" or "Apollo"

\section{Package}
\label{sec:org51ee1d8}
\subsection{Basics on packages}
\label{sec:orgf2f585d}
\begin{minted}[]{latex}
\NeedsTeXFormat{LaTeX2e}
\ProvidesClass{reporti}[Template LaTeX based on Texas Instruments note application documentation.]
\LoadClass[10pt, a4paper, twoside]{article}
\end{minted}

\subsection{Class options}
\label{sec:orgbe6a759}
\begin{minted}[]{latex}
\DeclareOption{draft}{
    \PassOptionsToPackage{draft}{graphicx}
    \AtEndOfClass{\usepackage{draftwatermark}\SetWatermarkText{DRAFT}}
}
\DeclareOption*{\PassOptionsToClass{\CurrentOption}{article}}
\ProcessOptions\relax
\end{minted}

\subsection{Packages}
\label{sec:org9596467}
\begin{minted}[]{latex}
\RequirePackage{xparse}        % Para comandos avanzados
\RequirePackage{geometry}      % Configuración de márgenes
\RequirePackage{fancyhdr}      % Encabezados/pies de página
\RequirePackage{titlesec}      % Estilos de secciones
\RequirePackage{graphicx}      % Manejo de imágenes
\RequirePackage[spanish]{babel}% Localización
\RequirePackage{hyperref}      % Hipervínculos
\RequirePackage{fontspec}      % Fuentes modernas
\RequirePackage{datetime2}     % Manejo de fechas profesional
\RequirePackage{xcolor}        % Manejo de colores
\RequirePackage{etoolbox}      % Utilidades de macros
\RequirePackage{tabularx}
\RequirePackage{float}
\RequirePackage{booktabs}
\RequirePackage{multirow}
\RequirePackage{tocloft}
\RequirePackage[tableposition=above]{caption}
\RequirePackage[figure,table]{totalcount}
\RequirePackage{xstring} % Add near the top of the file
\RequirePackage{underscore}
\RequirePackage{longtable}
\RequirePackage{wrapfig}
\RequirePackage{rotating}
\RequirePackage[normalem]{ulem}
\RequirePackage{amsmath}
\RequirePackage{amssymb}
\RequirePackage{capt-of}
\RequirePackage[outputdir=./build]{minted}
\usemintedstyle{emacs}


% Configuración de datetime2 para español
\DTMsetup{useregional=numeric}

\graphicspath{{figures/}{figures/logo/}}
\end{minted}

\subsection{Doument Commands}
\label{sec:orge967a07}
\begin{minted}[]{latex}
\newcommand{\@metadata}{} % Registro de metadatos

\NewDocumentCommand{\autor}{m}{%
    \def\@autor{#1}%
    \listadd{\@metadata}{Autor: #1}%
}
\NewDocumentCommand{\fecha}{O{\DTMtoday}}{%
    \def\@fecha{#1}%
    \listadd{\@metadata}{Fecha: #1}%
}
\NewDocumentCommand{\summary}{m}{%
    \def\@summary{#1}%
    \ifx\@summary\@empty\else
        \gappto\@afterabstract{\@printsummary}%
    \fi
}
\NewDocumentCommand{\subtitle}{m}{%
    \def\@subtitle{#1}%
    \listadd{\@metadata}{Subtítulo: #1}%  % Opcional: para mostrar en metadata
}
\NewDocumentCommand{\type}{m}{%
    \def\@type{#1}%
    \listadd{\@metadata}{Type: #1}%  % Opcional: para mostrar en metadata
}

\NewDocumentCommand{\projectcode}{m}{%
    \def\@projectcode{#1}%
    \listadd{\@metadata}{Projectcode: #1}%  % Opcional: para mostrar en metadata
}
\NewDocumentCommand{\notes}{m}{%
    \def\@notes{#1}%
    \listadd{\@metadata}{Notes: #1}%  % Opcional: para mostrar en metadata
}
\NewDocumentCommand{\contact}{m}{%
    \def\@contact{#1}%
    \listadd{\@metadata}{Contact: #1}%  % Opcional: para mostrar en metadata
}
\NewDocumentCommand{\toc}{m}{%
    \def\@toc{#1}%
    \listadd{\@metadata}{TOC: #1}%  % Opcional: para mostrar en metadata
}
\NewDocumentCommand{\logo}{m}{%
    \def\@logo{#1}%
    \listadd{\@metadata}{LOGO: #1}%  % Opcional: para mostrar en metadata
}

% ========================
% 3. Diseño de Página
% ========================
\renewcommand{\sectionmark}[1]{\markright{#1}}

\setlength{\headheight}{50pt} % 32.05278pt mínimo requerido + margen de seguridad
\geometry{top=3cm, bottom=2.5cm, left=1.83cm, right=1.83cm, footskip=1.2cm}

\fancypagestyle{mainstyle}{
  \fancyhf{}
  \renewcommand{\headrulewidth}{2pt}
  \renewcommand{\footrulewidth}{1.5pt}
  \ifdefvoid{\@logo}{}{%
    \fancyhead[RE,LO]{\includegraphics[height=2em]{\@logo}\par
      \href{www.rhyloo.com}{www.rhyloo.com}\par\par}
    }
  \fancyhead[LE,RO]{\nouppercase{\rightmark}}
  \fancyfoot[LE]{{\fontsize{8}{35}\selectfont \thepage \hspace{0.8cm} \itshape\@title}}
  \ifdefvoid{\@contact}{\fancyfoot[RE,LO]{{\fontsize{8}{35}\selectfont \ifdefvoid{\@projectcode}{}{\@projectcode}}}}{%
    \fancyfoot[RE,LO]{{\fontsize{8}{35}\selectfont \ifdefvoid{\@projectcode}{}{\@projectcode}\\\itshape\@contact}}
    }
  \fancyfoot[RO]{{\fontsize{8}{35}\selectfont  \itshape\@title \hspace{0.8cm} \thepage}}
}

\fancypagestyle{tocstyle}{
  \fancyhf{}
  \renewcommand{\headrulewidth}{2pt}
  \renewcommand{\footrulewidth}{1.5pt}
  \ifdefvoid{\@logo}{}{%
    \fancyhead[RE,LO]{\includegraphics[height=2em]{\@logo}\par
    \href{www.rhyloo.com}{www.rhyloo.com}}
    }
  \fancyhead[LE,RO]{}
  \fancyfoot[LE]{{\fontsize{8}{35}\selectfont \thepage \hspace{0.8cm} \itshape\@title}}
  \ifdefvoid{\@contact}{\fancyfoot[RE,LO]{{\fontsize{8}{35}\selectfont \ifdefvoid{\@projectcode}{}{\@projectcode}}}}{%
    \fancyfoot[RE,LO]{{\fontsize{8}{35}\selectfont \ifdefvoid{\@projectcode}{}{\@projectcode}\\\itshape\@contact}}
    }
  \fancyfoot[RO]{{\fontsize{8}{35}\selectfont  \itshape\@title \hspace{0.8cm} \thepage}}
}

% --- Redefinir marcas para ignorar subsecciones ---
\renewcommand{\sectionmark}[1]{\markright{#1}} % Actualiza solo \rightmark
\renewcommand{\subsectionmark}[1]{} % Subsecciones no modifican los headers


\pagestyle{mainstyle}


\fancypagestyle{titlepagestyle}{
  \renewcommand{\headrulewidth}{0pt}
  \renewcommand{\footrulewidth}{1.5pt}
  \fancyhf{}
  \fancyfoot[LE]{{\fontsize{8}{35}\selectfont \thepage \hspace{0.8cm} \itshape\@title}}
  \ifdefvoid{\@contact}{  \fancyfoot[RE,LO]{{\fontsize{8}{35}\selectfont \ifdefvoid{\@projectcode}{}{\@projectcode}}}}{
  \fancyfoot[RE,LO]{{\fontsize{8}{35}\selectfont \ifdefvoid{\@projectcode}{}{\@projectcode}\\\itshape\@contact}}}
  \fancyfoot[RO]{{\fontsize{8}{35}\selectfont  \itshape\@title \hspace{0.8cm} \normalfont\thepage}}
}


% ========================
% 4. Document Font
% ========================
\setmainfont{FreeSans}

% ========================
% 4. Estilos de Títulos
% ========================
\setlength{\voffset}{10pt} % Ajusta según necesidad para evitar warnings
\setlength{\headsep}{5pt} % Ajusta según necesidad para evitar warnings

% Definir formato de títulos
\titleformat{\section}
  {\large\bfseries} % Formato del texto
  {\thesection.\hspace{2em}}   % Etiqueta: Número + 4em de espacio
  {0pt}                        % Separación entre etiqueta y título
  {}                           % Código antes del título

\titleformat{\subsection}
  {\large\itshape\bfseries}
  {\thesubsection.\hspace{1.25em}} % 3em de espacio
  {0pt}
  {}

\titleformat{\subsubsection}
  {\bfseries}
  {\hspace{3.5em}} % 2em de espacio
  {0pt}
  {}

% Ajustar espaciado vertical (opcional)
\titlespacing{\section}{0pt}{12pt}{6pt}
\titlespacing{\subsection}{0pt}{12pt}{6pt}
\titlespacing{\subsubsection}{0pt}{12pt}{6pt}



\newlength{\originalparskip}
% ========================
% 6. Cover
% ========================
\NewDocumentCommand{\cover}{O{width=0.8\textwidth} O{}}{%
  \begingroup % Grupo local para cambios de espaciado
  % Save the original parskip value
  \setlength{\originalparskip}{\parskip}
  \renewcommand{\baselinestretch}{0.4} % Ajuste principal (20% más de espacio)
  \renewcommand{\parskip}{\originalparskip} % Ajuste principal (20% más de espacio)
  \begin{titlepage}
  \thispagestyle{titlepagestyle} % <--- Aplica el footer
    % --------------------------
    % Cabecera con logo y datos institucionales
  % --------------------------
  \ifdefvoid{\@logo}
            {\vspace*{4em}}{}
  \noindent\begin{minipage}[t]{0.4\textwidth}
  \ifdefvoid{\@logo}
            {}
            {\includegraphics[#1]{\@logo}}
    %% \vspace{-1.5\baselineskip} % Alineación precisa
    \end{minipage}%
    \hfill
    \begin{minipage}[t]{0.6\textwidth}
      \vspace{-2\baselineskip} % Alineación precisa
      \raggedleft
      \ifdef{\@type}{\fontsize{14}{35}\selectfont\itshape\@type}{}\par
      \ifdef{\@projectcode}{\fontsize{9}{35}\selectfont\itshape\@projectcode}{}

    \end{minipage}
    
    \vspace{0.5\baselineskip} % Alineación precisa

    \begin{flushright}
      {\fontsize{18}{35}\selectfont\bfseries\itshape\@title}
    \end{flushright}
    \vspace*{-1\baselineskip} % Reduce espacio posterior
        {\rule{\textwidth}{2pt}}

        \noindent\begin{minipage}{0.7\textwidth}
        \raggedright
        \ifdef{\@author}{%
          {\fontsize{10}{35}\selectfont\itshape\@author}
        }{}
        \end{minipage}
        \hfill
        \begin{minipage}{0.25\textwidth}
        \raggedleft
        \ifdef{\@subtitle}{%
          {\fontsize{10}{35}\selectfont\itshape\@subtitle}
        }{}
        \end{minipage}

        % --------------------------
        % Summary condicional
        % --------------------------
        \ifdefvoid{\@summary}{}{%
          \vspace{0.75cm}

          \hspace{1.75cm}
          \begin{minipage}{13.6cm}
            \begin{center}
              \noindent{\bfseries ABSTRACT}
              \end{center}
            {\fontsize{10}{35}\selectfont\@summary}\par
            {\rule{\textwidth}{1pt}}
          \end{minipage}
        }
        \ifstrequal{#2}{continue}{
          \tocloftpagestyle{titlepagestyle}
          \vspace{0.75cm}

          \hspace{1.75cm}
          \begin{minipage}{13.6cm}
            {\fontsize{9}{35}\selectfont
              \tableofcontents
              \vspace{1em}
              \iftotalfigures
              \listoffigures
              \fi
              \vspace{1em}
              \iftotaltables
              \listoftables
              \fi}
          \end{minipage}}{}
        \vfill
        \ifdefvoid{\@notes}{}{%
          \noindent\raggedright{\fontsize{8}{35}\selectfont\@notes\vspace{-1\baselineskip}}}
  \end{titlepage}
  \endgroup
  \clearpage
  \pagestyle{mainstyle}
}

\AddToHook{cmd/section/before}{%
  \clearpage
}
\tocloftpagestyle{tocstyle}
\setlength{\cftbeforesecskip}{0pt}
\renewcommand{\cftsecleader}{\cftdotfill{\cftdotsep}}
\renewcommand{\cftdotsep}{1}% Default is 4.5

\renewcommand{\cftaftertoctitleskip}{2em}
\renewcommand{\cftafterloftitleskip}{2em}
\renewcommand{\cftafterlottitleskip}{2em}

\renewcommand{\cftfigindent}{0pt}
\renewcommand{\cftfigpresnum}{\figurename~}
\renewcommand{\cftfigaftersnum}{.}
\setlength{\cftfignumwidth}{5em}
\renewcommand{\thefigure}{\arabic{section}-\arabic{figure}}
\renewcommand{\cftsecpagefont}{\normalsize}


% =============================================
% List of Tables Customization
% =============================================
% 1. Set table numbering format: section-table (e.g., 1-1)
\renewcommand{\thetable}{\arabic{section}-\arabic{table}}

% 2. Configure table entries in List of Tables
\renewcommand{\cfttabindent}{0pt}               % Remove indentation
\renewcommand{\cfttabpresnum}{\tablename~}      % Prefix: "Table "
\renewcommand{\cfttabaftersnum}{.}              % Suffix: "."
\setlength{\cfttabnumwidth}{5em}                % Width for table numbers

% 3. (Optional) Set section page numbers in TOC
\renewcommand{\cftsecpagefont}{\normalsize}     % Normal font for section page numbers


% ========================
% 9. Configuración de Hipervínculos
% ========================
\hypersetup{
    colorlinks = true,
    linkcolor = blue!70!black,
    urlcolor = blue!70!black,
    citecolor = green!60!black
}

\setlength{\parindent}{0pt}
\setlength{\baselineskip}{50pt}
\renewcommand{\baselinestretch}{1.2} % Ajuste principal (20% más de espacio)


\makeatletter
\apptocmd{\@afterheading}{
  \vspace{-1.2\baselineskip}
  \setlength{\parskip}{12pt}
  }{}{}
\makeatother

\addto\captionsspanish{%
  \renewcommand{\contentsname}{\centering {\hfill \bfseries\normalsize Contenido\hfill}}
  \renewcommand{\tablename}{Tabla}
\renewcommand{\listfigurename}{\hfill\bfseries\normalsize Lista de figuras\hfill}
\renewcommand{\listtablename}{\hfill\bfseries\normalsize Lista de tablas\hfill}
%% \renewcommand{\lstlistlistingname}{\hfill\bfseries\normalsize Lista de códigos\hfill}
}
%% ---------------------------------------
%% TODO
%% ---------------------------------------
%% Si no hay código eliminarlo y subir el feedback.
\end{minted}
\end{document}
